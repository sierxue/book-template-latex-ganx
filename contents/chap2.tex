% file: chap2.tex

\chapter{一个温柔的开始}
\label{ch:aGentleStart}

\begin{lizi}
  {\large \akai 预测木瓜是否味美:}
  \upshape {想象一下你刚刚抵达太平洋小岛,
你很快发现木瓜是当地饮食的一个重要组成部分。
但是,你从来没有尝过木瓜。你想通过学习去预测你在市场上看到的木瓜是否可口。
首先,你需要决定你的预测应该基于木瓜的什么\textbf{\gls{feature}}
(\gls{feature2})。
根据你以前吃水果形成的经验,你决定使用两个特征:木瓜的颜色和木瓜的柔软度。
作为\textbf{学习者}
(learner)的你,可以随机地抽取一些木瓜作为\textbf{训练样本} (training
sample), 记录下每一个木瓜的颜色和柔软度,然后试吃以判定木瓜是否美味。
根据这些数据,你可以生成一个\textbf{预测规则} (prediction rule)。
我们的第一步,是建立一个数学模型来刻画这样的学习问题。}
\end{lizi}

\section{一个统计学习的数学模型}

\subsection{统计学习框架:}

这个统计框架 (statistical learning framework)包含四部分: \textbf{输入}
(input)、\textbf{输出} (output)、\textbf{数据生成模型} (data-generation
model)、 以及\textbf{成效} (measure of success)。

\subsubsection{学习者的输入}

\begin{enumerate}

\item
  \textbf{样本空间} (domain space, or domain set, or instance space):
  一个任意的集合,通常表示为\(\mathcal{X}\),它是我们想要研究的对象的集合。
  在上面的例子中,\(\mathcal{X}\)就是所有的木瓜的集合。
  通常\(\mathcal{X}\)中的一个\textbf{样本}
  (sample),通常表示为一个小写的\(x\), 可以由一个\textbf{特征向量}
  (feature vector)来表示,比如木瓜的颜色和柔软度。
  如果用一个值表示颜色,一个值表示柔软度,那么\(x\)可以用一个二元数组来表示。
  我们也称一个样本为一个\textbf{示例}
  (instance),\(\mathcal{X}\)为\textbf{示例空间} (instance space)。
\item
  \textbf{标记空间} (label space or label
  set):通常表示为\(\mathcal{Y}\)。
  对于本例,\(\mathcal{Y}\)为二值集合,通常表示为\(\{0,1\}\):1代表味美,0代表非味美。
\item
  \textbf{训练数据} (training data):通常表示为
  \[S = \left(\left(x_1,y_1\right),\ldots,\left(x_m,y_m\right)\right)\]
  其中,\(x_i\),
  \(i=1,\ldots,m\),为第\(i\)个样本,\(y_i\)为对第\(i\)个样本所做的标记。
  我们称\(\left(x_i,y_i\right)\)为第\(i\)个\textbf{样例}(example)。
  简单地说,训练数据就是一组使按顺序排列的样例。\footnote{示例和样例的区别是什么,他们对应的英文单词分别是什么?}
\end{enumerate}

\subsubsection{学习者的输出}

通过对样例的学习,学习者输出一个预测准则,\[h: \mathcal{X} \rightarrow \mathcal{Y}\]。
我们称这个函数为一个\textbf{预测器} (predictor),或是\textbf{假设}
(hypothesis),或是\textbf{分类器} (classifier)。
这个预测器可以用来标记新样本。在本例中,预测器可以用来标记市场上的木瓜是否味美。
根据训练数据,学习者可以构造某种学习算法来得到一个预测器。

\subsubsection{数据生成模型}

首先,我们假定样本来自于某个概率分布\(\mathcal{D}\)。
在本例中,学习者对\(\mathcal{D}\)\emph{一无所知},并且\(\mathcal{D}\)可以是任意的概率分布。
在本例中,我们假设有一个``正确''的预测器\(f\)。
预测器\(f\)对于学习者来说也是未知的。事实上,学习者梦寐以求的正是这个预测器。
总的说来,\(S\)中的任意一个样例\(\left(x_i,y_i\right)\)是这样产生的:

\begin{enumerate}

\item
  按照\(\mathcal{D}\)抽样产生一个样本\(x_i\);
\item
  用正确的预测器来标记\(x_i\),即\(y_i=f(x_i)\).
\end{enumerate}

\subsubsection{成效}

学习者通过对训练数据的学习,生成一个预测器\(h\)后,怎么来评判这个预测器的好坏呢?
一个顺理成章的标准是判断这个预测器犯错误的概率的大小。数学上,我们可以表示为:
\[
\begin{aligned}
L_{\mathcal{D},f}(h) \overset{\underset{\mathrm{def}}{}}{=} \mathbb P_{x \sim D}\left[h(x) \neq f(x)\right] \overset{\underset{\mathrm{def}}{}}{=} \mathcal{D}\left(\left\{x:h(x) \neq f(x)\right\}\right)
\end{aligned}
\] 我们称\(L_{\mathcal{D},f}(h)\)为\textbf{预测错误} (error of a
predictor 或者 error of a classifier 或者 generalization error 或者 risk
或者 true error)。 注意\(L_{\mathcal{D},f}(h)\)中的下标
\(\mathcal{D},f\) 表示这个预测错误的度量 (measure)
是基于样本分布\(\mathcal{D}\)及正确的预测器\(f\)。 如果上下文的语境中
\(\mathcal{D},f\) 很清楚的话,我们有时为简洁的缘故会省略掉下标。

\subsubsection{思考和练习}

\begin{enumerate}

\item
  在本例中,输入、输出、和数据生成模型分别是什么?
\item
  学习者所拥有的信息是什么?\footnote{简单说来,学习者所有信息就是训练数据\(S\)。在本例中,最初,学习者不知道木瓜颜色和柔软度的概率分布,也不知道具有某种颜色和某种柔软度的木瓜口味如何。学习者对木瓜的了解来源于\(S\)。}
\end{enumerate}
